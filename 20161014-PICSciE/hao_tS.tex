\documentclass[serif,14pt]{beamer}
\usepackage[T1]{fontenc}
\usepackage{fourier}
\usepackage{amssymb}
\usepackage{amsmath}
\usepackage{bm}
\usepackage{bbm}
\usepackage{pifont}
\usepackage{xfrac}
\usepackage{multirow,bigdelim}
\usepackage{datetime}
\usepackage{tikz}
\usepackage[caption=false]{subfig}


\usepackage{ulem}

\usepackage{pgfpages}
%\setbeameroption{show notes}
%\setbeameroption{show notes on second screen=left}
%\setbeamertemplate{note page}[plain]

\usepackage{mathtools} %for underbraces

\yyyymmdddate

\newcommand{\E}{\mathrm{E}}

\mode<presentation>
{ \usetheme{Frankfurt} 
  \usecolortheme{seagull} }
\usepackage{graphicx}

\setbeamertemplate{caption}[numbered]
\setbeamertemplate{navigation symbols}{}
\setbeamertemplate{headline}{}

\setbeamertemplate{section page}
{
    \begin{centering}
    \begin{beamercolorbox}[sep=12pt,center,rounded=true,shadow=false]{part title}
    \usebeamerfont{section title}\sc\insertsection\par
    \end{beamercolorbox}
    \end{centering}
}

\setbeamertemplate{title page}[default][colsep=-4bp,rounded=true]


\title{ Scaling models of genetic variation to millions of humans }
\author{ Wei Hao \\ {\footnotesize Lewis-Sigler Institute for Integrative Genomics}}
\date{\today}

% mention John STorey
% mention that I'm LSI
% mention he's CSML
% mention collab w blei et al

% reduce bio intro to 5min fewer slides

\begin{document}

\setlength{\parskip}{8pt}

\begin{frame}
  \titlepage
\end{frame}

\begin{frame}

[DNA PICTURE]

genetic information is encoded in DNA; humans share 99+ percent of DNA; yet humans exhibit a great variety of complex traits; for example, height or disease suspectability.
\end{frame}

\begin{frame}
Single Nucleotide Polymorphism (SNP):

[SNP PICTURE]

genetic variation is key; most common type of genetic variation is a SNP; SNPs are where a single base pair in DNA seq differs; 
\end{frame}

\begin{frame}

[INHERITANCE PICTURE]

genetic variants are passed from parents to offspring; you have a version of each chromosome from each of your parents;
\end{frame}

\begin{frame}

[GLOBE PICTURE]

thus, genetic variation is affected by migration; and the general course of human history
\end{frame}

\begin{frame}

[DARWIN PICTURE OR DAVID STERN BLACK SCREEN?]

as well as evolutionary forces such as selection
\end{frame}

\begin{frame}

[DAVID STERN BLACK SCREEN?]

thus, modeling genome wide variation is important step in understanding the role that genetics plays.
\end{frame}

\begin{frame}

[SNPS PICTURE]

describe data, 0's 1's and 2's; two dimensions- samples and locations in genome; give example of scale; discuss how this is a opportunity to model dependence b/w indivs due to high-dimensionality
\end{frame}

\begin{frame}

[ANCESTRAL POPULATIONS PICTURE]

start describing model; show ancestral poops using solid color bars
\end{frame}

\begin{frame}

[ANCESTRAL POPULATIONS PICTURE 2]

show individuals as linear combinations (admixture) of color bars
\end{frame}

\begin{frame}

[MARKER PICTURE]

at each variant, ancestral populations have an associated frequency for how common this variant is; bc each individuals genome is composed of contributions from each ancestral pop
\end{frame}

\begin{frame}

MATH/MATRIX SLIDE

this setup lends itself easily to a Bayesian model; 
\end{frame}

\begin{frame}

[ROSENBERG 2002 FIGURE]

end result looks like this; posterior for the admixture proportions; global dataset; scale
\end{frame}

\begin{frame}

[PSD CITATIONS FIGURE]

this admixture model is wildly popular in population genomics; discuss PSD citations; discuss similarities to LDA for NLP problems
\end{frame}

\begin{frame}

[PAPER TITLES FIGURE]

many different ways to infer: structure model, admixture, faststructure; describe each
\end{frame}

\begin{frame}

[NEXT GEN DATA PAPER FIGURES]

fundamental weakness is these methods are not scalable- show 23andme, GWAS, TGP p3 papers
\end{frame}

\begin{frame}

[????]

discuss that we'll use SVI to fit this model
\end{frame}

\begin{frame}

[TS SCHEMATIC]

discuss TS algo; discuss sampling; discuss local v global
\end{frame}

\begin{frame}

[SIMULATION FIGURE]

we can beat everyone else up to millions
\end{frame}


\begin{frame}

[????]

some sort of closing?
\end{frame}

\begin{frame}
Thanks co-authors
\end{frame}

\begin{frame}
Software: \url{https://github.com/premgopalan/terastructure}

Funding:
\begin{itemize}
\item NIH R01 HG006448, Models and Methods for Population Genomics 
\item NIH P50 GM071508, Center for Quantitative Biology 
\end{itemize}
\end{frame}

\end{document}
